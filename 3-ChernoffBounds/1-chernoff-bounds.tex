\section{Chernoff Bounds}
\subsection{Case for $\P{X > (1 + \delta) \mu}$}
Givet uafhængige Poisson trials $X_1, \dots, X_n$, dvs. for $i = 1, \dots, n$, lad $X_i$ være en indikatorvariabel med $\P{X_i = 1} = p_i$ hvor $0 < p_i < 1$. Lad
$$
\delta > 0
\quad\quad\quad\quad
X = \sum_{i=1}^n X_i
\quad\quad\quad\quad
\mu = \mu_X = \sum_{i=1}^n p_i
$$
Da gælder:
\begin{align*}
  \P{X > (1+\delta) \mu} < \pfrac{e^\delta}{(1+\delta)^{1 + \delta}}^\mu
\end{align*}


\subsubsection{Bevis}
Lad $t > 0$ (fastlægges senere til $t = \ln(1 + \delta)$). Da gælder:
\begin{align}
  \P{X > (1+\delta) \mu}
  &= \P{e^{tX} > e^{t(1+\delta) \mu}} \label{eq:i-ete} \\
  &< \frac{\E[e^{tX}]}{e^{t(1+\delta) \mu}} \label{eq:bruger-markov} \\
  &= \frac{\E[\prod_{i=1}^n e^{t X_i}]}{e^{t(1+\delta) \mu}} \label{eq:stok-var} \\
  &= \frac{\prod_{i=1}^n \E[e^{t X_i}]}{e^{t(1+\delta) \mu}} \label{eq:uafhaang} \\
  &= \frac{\prod_{i=1}^n \p{p_i e^t + (1-p_i) e^0 }}{e^{t(1+\delta) \mu}} \label{eq:brug-exp} \\
  &= \frac{\prod_{i=1}^n \p{1 + p_i (e^t - 1)}}{e^{t(1+\delta) \mu}} \label{eq:reducer} \\
  &\leq \frac{\prod_{i=1}^n e^{p_i (e^t - 1)}}{e^{t(1+\delta) \mu}} \label{eq:brug-1+x-regel} \\
  &= \frac{e^{\p{\sum_{i=1}^n p_i (e^t - 1)}}}{e^{t(1+\delta) \mu}} \label{eq:prod-til-sum} \\
  &= \frac{e^{\mu (e^t - 1)}}{e^{t(1+\delta) \mu}} \label{eq:beregn-sum} \\
  &= \frac{e^{\delta \mu}}{(1 + \delta)^{(1 + \delta) \mu}} \label{eq:indsaat-t} \\
  &= \pfrac{e^\delta}{(1 + \delta)^{1 + \delta}}^\mu \label{eq:reducer-potens}
\end{align}

I \cref{eq:i-ete} opløfter vi $e$ med vores udtryk, hvorved uligheden stadig gælder. Dette er blot noget som skal memoreres til eksamen.\\
I \cref{eq:bruger-markov} bruger vi Markovs ulighed.\\
I \cref{eq:stok-var} bruger vi definitionen for $X$, og bestemmer produktet i stedet for summen da den står i en potens.
I \cref{eq:uafhaang} benytter vi at vores stokastiske variable er uafhængige, og derfor kan vi flytte vores forventning ind i produktet.\\
I \cref{eq:brug-exp} benytter vi definitonen på forventningsværdien og egenskaberne for Poisson trials.\\
I \cref{eq:reducer} reducerer vi blot udtrykket.\\
I \cref{eq:brug-1+x-regel} benytter vi reglen $1 + x \leq e^x \ \forall x \in \R$.\\
I \cref{eq:prod-til-sum} har vi, at produktet af en masse udtryk med samme base svarer til summen af alle potenserne.\\
I \cref{eq:beregn-sum} benytter vi vores definition af $\mu$.\\
I \cref{eq:indsaat-t} indsætter vi vores $t = \ln(1 + \delta)$ så vores udtryk bliver pænt.\\
I \cref{eq:reducer-potens} sætter vi blot $\mu$ uden for en parentes for brøken.



\subsection{Case for $\P{X < (1 - \delta) \mu}$}
Givet uafhængige Poisson trials $X_1, \dots, X_n$, dvs. for $i = 1, \dots, n$, lad $X_i$ være en indikatorvariabel med $\P{X_i = 1} = p_i$ hvor $0 < p_i < 1$. Lad
$$
0 < \delta \leq 1
\quad\quad\quad\quad
X = \sum_{i=1}^n X_i
\quad\quad\quad\quad
\mu = \mu_X = \sum_{i=1}^n p_i
$$
Da gælder:
\begin{align*}
  \P{X < (1 - \delta) \mu} < e^{- \frac{\delta^2 \mu}{2}}
\end{align*}


\subsubsection{Bevis}
Lad $t > 0$ (fastlægges senere til $t = - \ln(1 - \delta)$). Da gælder:
\begin{align}
  \P{X < (1 - \delta) \mu}
  &= \P{e^{- t X} > e^{-t (1 - \delta) \mu} } \label{eq:ref1} \\
  &< \frac{\E[e^{-t X}]}{e^{-t (1 - \delta) \mu}} \nonumber \\
  &= \frac{\prod_{i=1}^n \E[e^{-t X_i}]}{e^{-t (1 - \delta) \mu}} \nonumber \\
  &= \frac{\prod_{i=1}^n \p{p_i e^{-t} + (1 - p_i) e^0}}{e^{-t (1 - \delta) \mu}} \nonumber \\
  &= \frac{\prod_{i=1}^n \p{1 + p_i (e^{-t} -1)}}{e^{-t (1 - \delta) \mu}} \nonumber \\
  &\leq \frac{\prod_{i=1}^n e^{p_i (e^{-t} - 1)}  }{e^{-t (1 - \delta) \mu}} \nonumber \\
  &= \frac{e^{\mu(e^{-t} - 1)}  }{e^{-t (1 - \delta) \mu}} \label{eq:ref2} \\
  &= \frac{e^{- \delta \mu}  }{ (1 - \delta)^{(1 - \delta) \mu} } \label{eq:indsaat-neg-t} \\
  &= \pfrac{e^{-\delta}}{(1 - \delta)^{1 - \delta}}^\mu \nonumber \\
  &= \pfrac{e^{-\delta}}{e^{(1 - \delta) \ln (1 - \delta)}}^\mu \label{eq:omskriver} \\
  &< \pfrac{e^{- \delta}}{e^{- \delta + \frac{1}{2} \delta^2}}^\mu \label{eq:mclaurin} \\
  &= e^{- \frac{1}{2} \delta^2 \mu} \label{eq:pot-regne-regler}
\end{align}

Fra \cref{eq:ref1} til \cref{eq:ref2} er det stort set det samme vi gjorde som før. I \cref{eq:indsaat-neg-t} indsætter vi så $t = - \ln(1 - \delta)$ (dette gælder egentlig kun for $\delta < 1$, men udtrykket kan dog let bevises når $\delta = 1$). I \cref{eq:omskriver} omskriver vi blot for at få noget med $e$ i nævneren. I \cref{eq:mclaurin} benytter vi McLaurin expansion, som medfører at $(1 - \delta) \ln(1 - \delta) > - \delta + \frac{1}{2} \delta^2$. Og endelig i \cref{eq:pot-regne-regler} er det blot simple potensregneregler.

\subsubsection{McLaurin Expansion}
Givet en funktion $f(x) : \R \to \R$, som er vilkårligt ofte differentiabel i $x = 0$, da gælder:
$$
f(x) = \sum_{i=0}^n \frac{f^{(i)} (0)}{i!} x^i
$$

For vores konkrete case ovenfor betyder det:
$$
(1 - \delta) \ln(1 - \delta)
= - \delta + \frac{1}{2} \delta^2 + \frac{1}{6} \delta^3 + \frac{1}{12} \delta^4 + \frac{1}{20} \delta^5 + \dots > - \delta + \frac{1}{2} \delta^2
$$
