\section{Randomized Selection}
\subsection{Algoritme}
Tager en uordnet liste $S$ med $n$ elementer og $k \in \{1, ..., n\}$ som input og returnerer det $k$'te mindste element i $S$.\\

I følgende bruger vi notationen:\\
$L_{(i)}$: Det $i$'te mindste element i en liste $L$.\\
$r_L(a)$: Rank af element $a$ i listen $L$.\\


\texttt{LazySelect}$(S, k)$
\begin{enumerate}
  \item Vælg $n^{3/4}$ elementer uafhængigt og uniformt tilfældigt fra $S$. Lad $R$ være en multimængde af valgte elementer.
  \item Sorter $R$ (dette ødelægger ikke vores køretid da $R$ relativt set er lille).
  \item Lad $x = kn^{-1/4}$ og
  \begin{align*}
    l = \max\{\floor{x - \sqrt{n}}, 1\}
    \quad\quad & \quad\quad
    h = \min\{\ceil{x + \sqrt{n}}, n^{3/4} \} \\
    a = R_{(l)}
    \quad\quad & \quad\quad
    b = R_{(h)}
  \end{align*}
  Bestem rank $r_S(a)$ og $r_S(b)$.
  \item Antag $k \in \square{ n^{1/4}, n-n^{1/4} }$ og lad $P = \{ y \in S | a \leq y \leq b \}$.\\
  Hvis $S_{(k)} \notin P$ eller $|P| > 4n^{3/4}+2$, gentag ovenstående.\\

  $P$ kunne f.eks. løbende være blevet lavet mens vi bestemte rank for $a$ og $b$. Det er dette step, som bidrager med vores $2n$ sammenligninger.
  \item Sorter $P$ og returner $P_{(k - r_S(a) + 1)}$ .
\end{enumerate}

\subsection{Eksempel}
Her ses et eksempel, hvor vi har et array hvor $n = 40$ og vi ønsker at finde det 37'ende mindste element, $k = 37$. På figuren nedenfor udføres først del 1 og 2 af algoritmen, hvor $n^{3/4} \approx 16$. Antag at alle de elementer vi valgte ud tilfældigvis var unikke. Så får vi måske følgende:
\begin{figure}[H]
  \begin{center}
  \includegraphics[width=\textwidth]{lazy1.pdf}
  \end{center}
  %\caption{Array $s$, hvor $n^{3/4}$ elementer er valgt og sorteret $R$}
  \label{fig:lazy1}
\end{figure}

Herefter fås $x = kn^{-1/4} \approx 15$ og herved $l = 8$ og $h = 16$. Derved må $a$, som er det 8'te mindste element i $R$ og $b$, som er det 16'te mindste element i $R$, kunne ses som:
\begin{figure}[H]
  \begin{center}
  \includegraphics[width=\textwidth]{lazy2.pdf}
  \end{center}
  %\caption{$a$ og $b$ i sorteret $R$}
  \label{fig:lazy2}
\end{figure}

Herefter bestemmer vi rank af $a$ og $b$, $r_S(a)$ og $r_S(b)$, i forhold til den originale liste $S$. Samtidig laver vi ud fra dette vores nye liste $P$. Det er dette, som bidrager med vores $2n$ sammenligninger. Hvis vi tegner vores nye liste $P$, sammenholdt med $S$ hvor den var sorteret, kunne det se således ud:
\begin{figure}[H]
  \begin{center}
  \includegraphics[width=\textwidth]{lazy3.pdf}
  \end{center}
  %\caption{caption}
  \label{fig:lazy3}
\end{figure}

Her svarer de markerede grønne felter til de 16 felter vi valgte tilfældigt i $S$ til at starte med, men nu hvor den (blot på figuren, ikke i algoritmen) er skrevet op sorteret for at kunne sammenligne med $P$, som bliver sorteret af algoritmen. Vi fandt her, at $r_S(a) = 17$ og $r_S(b) = 39$.\\

Da vi ikke fejler nogle af vores test i step 4 af algoritmen fortsætter vi til step 5. Vi får herved $P_{(k - r_S(a) + 1)} = P_{(37 - 17 + 1)} = P_{(21)}$, som svarer til $S_{(37)}$ hvilket netop var hvad vi skulle bestemme (markeret med gult på figuren).\\

På et større taleksempel ville det typisk relativt set være en mindre del af $R$ der ville blive valgt ud.



\subsection{Sandsynlighed for succes i første iteration}
Med sandsynlighed $1 - O(n^{-1/4})$ er der succes i første iteration af linje 4 og dermed laver algoritmen kun $2n + o(n)$ sammenligninger.\\

\textbf{Bevis}:\\
Vi har tre mulige fejltyper:\\
Fejltype 1: $S_{(k)} < a$\\
Fejltype 2: $S_{(k)} > b$\\
Fejltype 3: $|P| > 4 n^{3/4} + 2$.\\\\

Vi fokuserer på fejltype 1. Det er en fejl vi får i følgende tilfælde:
\begin{figure}[H]
  \begin{center}
  \includegraphics[width=\textwidth]{fejltype1.pdf}
  \end{center}
  %\caption{caption}
  \label{fig:fejltype1}
\end{figure}




For alle vores udtrukne elementer $i = 1, ..., n^{3/4}$, lad
$
X_i =
\begin{cases}
	1 & \text{hvis det $i$'te sample $\leq S_{(k)}$}\\
	0 & \text{ellers}
\end{cases}
$\\

Da er $\E[X_i] = \P{X_i = 1} = \frac{k}{n}$, som det tydeligt ses på figuren da de første $1, ..., k$ elementer naturligvis vil være $\leq k$, og der er $n$ elementer i alt.

Lad $X = \sum_{i=1}^{n^{3/4}} X_i$. Så får vi den forventede værdi til:
$$
\mu_X = \sum_{i=1}^{n^{3/4}} \mu_{X_i} = \sum_{i=1}^{n^{3/4}} \frac{k}{n} = n^{3/4} * \frac{k}{n} = kn^{-1/4} = x
$$

I eksemplet på fejltypen ovenfor er $X = 0$ og $l = 1$.

Og da kan vi beregne:
\begin{align}
  \P{S_{(k)} < a} &= \P{X < l} \label{eq:x-less-l}  \\
  &= \P{X < \max \left\{ \floor{x - \sqrt{n}}, 1 \right\} } \nonumber\\
  &\leq \P{X \leq \floor{x - \sqrt{n}}} \label{eq:fjern-max} \\
  &\leq \P{X \leq x - \sqrt{n}} \nonumber\\
  &= \P{X \leq \mu_X - \sqrt{n}} \label{eq:p-simpel}
\end{align}

I \cref{eq:x-less-l} bruger vi, at $X$ er antal elementer (i en sorteret udgave af $S$) før eller ved placeringen af $S_{(k)}$. $l$ bestemmer hvilket element vi starter ved af de udtrukne. Derfor må fejltype 1 opstå, når antallet af udtrukne elementer der opfylder at være $\leq S_{(k)}$ er et mindre antal end vores ''start-index'' for de udtrukne elementer, $l$.\\
I \cref{eq:fjern-max} benytter vi, at før tog vi max af de to udtryk. Ved nu altid bare at tage ét af udtrykkene kan det kun potentielt blive mindre, hvorved sandsynligheden også kun potentielt bliver mindre.\\
\textbf{TODO: Hvad? Det giver jo ikke nogen mening at sandsynligheden potentielt bliver mindre.}\\

Nu bruger vi det lemma, at givet \emph{uafhængige} stokastiske variable $Y_1, ..., Y_m$, hvor vi lader $Y = \sum_{i=1}^m Y_i$, da er variansen $\sigma_Y^2 = \sum_{i=1}^m {\sigma_{Y_i}}^2$:
\begin{align}
  \sigma_X^2 &= \sum_{i=1}^{n^{3/4}} {\sigma_{X_i}}^2 \nonumber \\
  &= \sum_{i=1}^{n^{3/4}} \frac{k}{n} \p{ 1 - \frac{k}{n} } \label{eq:faa-paren} \\
  &\leq \sum_{i=1}^{n^{3/4}} \frac{1}{4} \label{eq:paren-to-max} \\
  &= \frac{n^{3/4}}{4} \nonumber \\
  &\Updownarrow \nonumber \\
  \sigma_X &= \frac{n^{3/8}}{2} \label{eq:std-dev}
\end{align}

I \cref{eq:faa-paren} bruger vi \textbf{TODO: Hvad... hvorfor er ${\sigma_{X_i}}^2 = k/n(1 - k/n)$?}\\
I \cref{eq:paren-to-max} benytter vi, at udtrykket $p(1 - p)$ er størst når $p = 1/2$ og da bliver $1/4$.
I \cref{eq:std-dev} tager vi kvadratroden og får herved standardafvigelsen.

Regner vi nu videre på \cref{eq:p-simpel} med denne viden får vi:
\begin{align}
  \P{S_{(k)} < a} &\leq \P{X \leq \mu_X - \sqrt{n}} \nonumber \\
  &= \P{-X \geq -\mu_X + \sqrt{n}} \nonumber \\
  &= \P{\mu_X - X \geq \sqrt{n}} \nonumber \\
  &\leq \P{|\mu_X - X| \geq \sqrt{n}}  \nonumber \\
  &= \P{|\mu_X - X| \geq 2n^{1/8} * \frac{n^{3/8}}{2}  } \label{eq:skriv-smart} \\
  &\leq \P{ |\mu_X - X| \geq 2n^{1/8} \sigma_X } \label{eq:skriv-til-sigma} \\
  &\leq \frac{1}{\p{2n^{1/8}}^2} \label{eq:brug-chebyshev} \\
  &\leq \frac{1}{4n^{1/4}} \nonumber \\
  &= O(n^{-1/4}) \nonumber
\end{align}

I \cref{eq:skriv-smart} skriver vi blot $\sqrt{n} = n^{1/2}$ på en smart måde for at kunne bruge $\sigma_X$.\\
I \cref{eq:skriv-til-sigma} indsætter vi så blot vores $\sigma_X$.\\
I \cref{eq:brug-chebyshev} bruger vi Chebyshevs ulighed.\\

Herved har vi altså vist, at sandsynligheden for at en fejl af fejltype 1 opstår i første iteration af algoritmen kun er $O(n^{-1/4})$, og at den herved kun laver $2n + o(n)$ sammenligninger ($2n$ fra at bestemme rank af $a$ og $b$ og $o(n)$ fra sorteringen af $R$ og $P$).\\

Beviset for fejltype 2 er symmetrisk. Og fejltype 3 vil vi ikke beskæftige os med.
